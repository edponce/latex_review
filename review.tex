\documentclass{ltxdockit}[2010/09/26]
\usepackage[utf8]{inputenc}
\usepackage[american]{babel}


\newcommand*\review{\sty{review}}

\titlepage{%
    title={The \review\ Package},
    subtitle={A \LaTeX\ Document Revision Package},
    url={http://www.ctan.org/pkg/review},
    author={Eduardo Ponce},
    email={edponce00@hotmail.com},
    revision={v1.0},
    date={2020/10/04}
}

\hypersetup{%
    pdftitle={The review Package},
    pdfsubject={A LaTeX Document Revision Package},
    pdfauthor={Eduardo Ponce},
    pdfkeywords={tex, latex, package, review, comments},
    pdfborder=0 0 0,
    pdfpagelayout=SinglePage,
    bookmarksnumbered=true,
    bookmarksopen=true,
    plainpages=false,
    colorlinks=true,
    linktoc=all
}


\begin{document}


\printtitlepage
\tableofcontents


\section{Introduction}
\label{intro}

The purpose of \review is simplify the revision and editing process when
working on a manuscript. There are several CTAN packages that provide similar
and related functionalities. We list these packages along with their
advantages and limitations compared to \review.

\begin{itemize}
\item https://www.ctan.org/pkg/changes
\item https://www.ctan.org/pkg/easyreview
\item https://www.ctan.org/pkg/minorrevision
\item https://www.ctan.org/pkg/ed
\item https://www.ctan.org/pkg/todonotes
\item https://www.ctan.org/pkg/fixmetodonotes
\item https://www.ctan.org/pkg/proofread
\end{itemize}

\subsection{About \review}
\label{about}

The \review\ package ...


\subsection{License}
\label{license}


\section{User Commands} \label{user}


\subsection{Definitions}
\label{user:def}

The following set of definitions make use of two optional parameters.
The optional \prm{ID} represents a unique user-defined identifier associated
with the reviewer suggesting the change. It is recommended to use a short
\prm{ID} such as the name initials in uppercase form. If no \prm{ID} is
provided then the change will not have a corresponding reviewer. The optional
\prm{color} is used to for coloring the changes in the text and the reviewr
\prm{ID}. The default color for is red.

\begin{ltxsyntax}

\cmditem{add}[ID]{text}[color]

Change that adds \prm{text} to the document.

\cmditem{delete}[ID]{text}[color]

Tags the selected text for removal.

\cmditem{replace}[ID]{old text}{new text}[color]

Change the replaces \prm{old text} with \prm{new text}.

\cmditem{note}[ID]{text}[color][options for \\todo macro]

The note command represents a regular note which uses a standalone box to place
the \prm{text}. Regular notes can be placed before/after paragraphs or inlined
with text. A regular box creates a new paragraph, so it appears by itself in
the current textwidth. Regular notes support responses as part of the
\prm{text} via the response command.

\cmditem{note}[ID]{text}[color]

\end{ltxsyntax}


\section{Reporting Issues}

The development code for \review\ is hosted on Github (provide url). This is the best place to log any
issues with the package.


\section{Revision History}

This revision history is a list of changes relevant to users of this package.
Changes of a more technical nature which do not affect the user interface or
the behavior of the package are not included in the list. If an entry in the
revision history states that a feature has been \emph{improved} or
\emph{extended}, this indicates a syntactically backwards compatible
modification, such as the addition of an optional argument to an existing
command. Entries stating that a feature has been \emph{modified} demand
attention. They indicate a modification which may require changes to existing
documents in some, hopefully rare, cases. The numbers on the right indicate the
relevant section of this manual.

\begin{changelog}

\begin{release}{1.0}{2020-10-04}
\item Initial public release
\end{release}

\end{changelog}


\end{document}
